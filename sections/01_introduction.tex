\section{Introduction}
Constraint enforcement---notably incompressibility ($\divergence \uu=0$) and magnetic Gauss's law ($\divergence \BB=0$)---is central to high-fidelity fluid and magnetohydrodynamic simulations. Classical projection and constrained-transport methods achieve stability but incur large Poisson solves that dominate direct numerical simulation (DNS) cost, particularly at high Reynolds numbers or fine resolutions. Emerging machine learning (ML) models provide fast surrogates, preconditioners, and learned correctors that promise to reduce wall-clock time while retaining physical fidelity. Yet the space of methods is fragmented, ranging from physics-informed neural networks (PINNs) to neural operators and hybrid solvers.

This document presents a structured first draft for an Overleaf-ready research paper. We define the constraint-preserving DNS problem, curate a taxonomy of numerical and ML approaches, and outline experiments and milestones toward an open, reproducible study. A placeholder figure illustrating the project pipeline is referenced in \cref{fig:pipeline}.

\subsection{Reading map}
Readers seeking background on classical algorithms can start with projection and constrained-transport literature \citep{chorin1968projection,brackbill1980fluid,kronbichler2020efficient}. For ML-oriented acceleration and constraint handling, foundational PINN work \citep{raissi2019physics} and modern divergence-free neural operators \citep{li2021fourier,patel2022physics} are key. Hybrid solver designs combining ML with traditional projections are surveyed in \citep{mishra2022physics,um2020solver,babaee2021physics}.
